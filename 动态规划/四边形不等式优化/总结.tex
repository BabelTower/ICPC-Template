\paragraph{一、DP时间复杂度}
$$
\text{时间复杂度} = \text{状态总数} \times \text{每个状态转移的状态数} \times \text{每次状态转移的时间}
$$

\paragraph{二、各类优化方式}
\subparagraph{1. 决策单调性}~{}
\par

四边形不等式的性质在一类 1D1D 动态规划中得出决策单调性,从而优化状态转移的复杂度。\par

\textbf{1D1D动态规划:} DP方程形如 $f_{r} = \min_{l=1}^{r-1}\{f_{l}+w(l,r)\}\quad\left(1 \leq r \leq n\right)$ 
,状态数为 $O(n)$ ,每一步决策量为 $O(n)$ 。\par

\textbf{决策单调性:} 设 $k_i$ 表示 $f[i]$ 转移的最优决策点,那么决策单调性可描述为 
$\forall i \leq j, k_i \leq k_j$。也就是说随着i的增大,所找到的最优决策点是递增态(非严格递增)。\par

定理: 若函数 $w(l,r)$ 满足四边形不等式,记 $h_{l,r}=f_l+w(l,r)$ 表示从 $l$ 转移过来的状态 $r$ , 
$k_{r}=\min\{l|f_{r}=h_{l,r}\}$ 表示最优决策点,则有$$\forall r_1 \leq r_2:k_{r_1} \leq k_{r_2}$$ \par

\textbf{四边形不等式:} 如果对于任意 $l_1\leq l_2 \leq r_1 \leq r_2$ ,
均有 $w(l_1,r_1)+w(l_2,r_2) \leq w(l_1,r_2) + w(l_2,r_1)$ 成立,
则称函数 $w$ 满足四边形不等式(简记为“交叉小于包含”)。
若等号永远成立,则称函数 $w$ 满足四边形恒等式。\par

我们根据决策单调性只能得出每次枚举 $l$ 时的下界,而无法确定其上界。因此,简单实现该状态转移方程仍然无法优化最坏时间复杂度。\par

\subparagraph{2. 决策单调性(分治)}~{}
\par

先考虑一种简单的情况,转移函数的值在动态规划前就已完全确定。即如下所示状态转移方程:

$$
f_{r} = \min_{l=1}^{r-1}w(l,r) \qquad\left(1 \leq r \leq n\right)
$$

在这种情况下,我们定义过程 $\textsf{DP}(l, r, k_l, k_r)$ 表示求解 $f_{l}\sim f_{r}$ 的状态值,
并且已知这些状态的最优决策点必定位于 $[k_l, k_r]$ 中,然后使用分治算法如 \textbf{单调性决策(分治)} 中所诉。 \par

使用递归树的方法,容易分析出该分治算法的复杂度为 $O(n\log n)$ ,
因为递归树每一层的决策区间总长度不超过 $2n$ ,而递归层数显然为 $O(\log n)$ 级别。 \par

\subparagraph{3. 决策单调性(二分栈)}~{}
\par

处理一般情况,即转移函数的值是在动态规划的过程中按照一定的拓扑序逐步确定的。
此时我们需要改变思维方式,由“确定一个状态的最优决策”转化为“确定一个决策是哪些状态的最优决策”。 \par

用栈维护单调的决策点,二分找到是哪些状态的最优决策,时间复杂度为 $O(n\log n)$。 \par

\subparagraph{4. 区间类(2D1D)动态规划}~{}
\par

在区间类动态规划(如石子合并问题)中,我们经常遇到以下形式的 2D1D 状态转移方程:

$$
f_{l,r} = \min_{k=l}^{r-1}\{f_{l,k}+f_{k+1,r}\} + w(l,r)\qquad\left(1 \leq l \leq r \leq n\right)
$$

直接简单实现状态转移,总时间复杂度将会达到 $O(n^3)$ ,但当函数 $w(l,r)$ 满足一些特殊的性质时,我们可以利用决策的单调性进行优化。\par

\textbf{区间包含单调性:}如果对于任意 $l \leq l' \leq r' \leq r$ ,均有 $w(l',r') \leq w(l,r)$ 成立,则称函数 $w$ 对于区间包含关系具有单调性。 \par
\textbf{四边形不等式:}如果对于任意 $l_1\leq l_2 \leq r_1 \leq r_2$ ,均有 $w(l_1,r_1)+w(l_2,r_2) \leq w(l_1,r_2) + w(l_2,r_1)$ 成立,
则称函数 $w$ 满足四边形不等式(简记为“交叉小于包含”)。若等号永远成立,则称函数 $w$ 满足 四边形恒等式 。\par

引理 1 :若满足关于区间包含的单调性的函数 $w(l, r)$ 满足四边形不等式,则状态 $f_{l,r}$ 也满足四边形不等式。\par

定理 1 :若状态 $f$ 满足四边形不等式,记 $m_{l,r}=\min\{k:f_{l,r} = g_{k,l,r}\}$ 表示最优决策点,则有

$$
m_{l,r-1} \leq m_{l,r} \leq m_{l+1,r}
$$

因此,如果在计算状态 $f_{l,r}$ 的同时将其最优决策点 $m_{l,r}$ 记录下来,那么我们对决策点 $k$ 的总枚举量将降为

$$
\sum_{1\leq l<r\leq n} m_{l+1,r} - m_{l,r-1} = \sum_{i=1}^n m_{i,n} - m_{1,i}\leq n^2
$$

\subparagraph{5. 满足四边形不等式的函数类}~{}
\par

为了更方便地证明一个函数满足四边形不等式,我们有以下几条性质:

性质 1 :若函数 $w_1(l,r),w_2(l,r)$ 均满足四边形不等式(或区间包含单调性),则对于任意 $c_1,c_2\geq 0$ ,函数 $c_1w_1+c_2w_2$ 也满足四边形不等式(或区间包含单调性)。

性质 2 :若存在函数 $f(x),g(x)$ 使得 $w(l,r) = f(r)-g(l)$ ,则函数 $w$ 满足四边形恒等式。当函数 $f,g$ 单调增加时,函数 $w$ 还满足区间包含单调性。

性质 3 :设 $h(x)$ 是一个单调增加的凸函数,若函数 $w(l,r)$ 满足四边形不等式并且对区间包含关系具有单调性,则复合函数 $h(w(l,r))$ 也满足四边形不等式和区间包含单调性。

性质 4 :设 $h(x)$ 是一个凸函数,若函数 $w(l,r)$ 满足四边形恒等式并且对区间包含关系具有单调性,则复合函数 $h(w(l,r))$ 也满足四边形不等式。