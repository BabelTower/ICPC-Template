\paragraph{概率 — 期望系统的定义}~{}
\par
概率 — 期望系统是一个带权的有向图。这个图中的点代表一个事件,而如果点 A 与点 B 之间有一条权为 p 的边,就表示 A 发生后, B 紧接着发生的概率是 p 。
初始的时候,有一个点(叫做初始点)代表的事件发生了,其他事件根据概率依次发生,每次只发生一个。
求其他各个事件发生次数的期望。记时间 A 的发生次数期望为 $E_{A}$ , A 到 B 的边权为 $P_{AB}$ \par
\textbf{限制}:
\begin{itemize}
    \item 对任意的 AB , $P_{AB} \leq 1$
    \item 对于任意点 A , $ \sum_{(A,B)\in E}{P_{AB}} \leq 1$,且对于系统中的所有点,至少有一个点使等号不成立。如果等号都成立的话这些事件将无穷无尽的发生下去,而概率 — 期望系统则变得没有意义(此时期望或者是无穷大,或者是 0 )。
    \item 不能有指向初始点的边,这是因为求解时我们把初始顶点的概率设为 1 。但是如果真的有这样的边,可以添加一个假点作为初始点,这个假点到真正的初始点有一条概率为 1 的边。
\end{itemize}
\paragraph{概率 — 期望系统的求解}~{}
\par
可以根据是否为DAG图,将问题分为两类。\par
有向无环图的概率 — 期望系统。这种系统是很简单的,因为它没有后效性,所以可以通过动态规划的方法在 O(E) 的时间内解决。许多使用动态规划解决的概率 — 期望问题都是基于这类系统的。\par
在有些问题中,我们需要解决更一般的概率 — 期望系统。这时图中含有圈,因而造成了后效性。\par

\paragraph{高斯消元解决后效性概率DP}~{}
\par
$$E_{A} = \sum_{(B,A)\in E}P_{BA} E_{B}$$ 
求线性方程组的解,我们更常用的稳定算法是高斯消元法,完全可以在这里使用。这样就得到了一种稳定而精确的解法:
首先根据概率 — 期望系统建立方程组,然后用高斯消元法去解,得到的结果就是我们要求的期望。这种算法的时间复杂度是 $O(n^3)$ 。

\paragraph{一种可以消除后效性的特例}~{}
\par
\textbf{限制}:
\begin{itemize}
    \item 图为有向无环图(线性递推、树),非DAG图需要用Tarjan算法缩点
    \item 存在无后效性可以转移的节点,即只从前置状态转移而来
    \item 每个状态只从常数个数的状态转移而来
\end{itemize}\par
\textbf{解法:}\par
根据DAG图的逆拓扑序,将所有节点的转移方程依次变为无后效性的。原理等价于高斯消元,