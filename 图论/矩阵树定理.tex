\paragraph{矩阵树定理}~{}
\\
对于生成树的计数,一般采用矩阵树定理(Matrix-Tree 定理)来解决。
\\
Matrix-Tree 定理的内容为:对于已经得出的基尔霍夫矩阵,去掉其随意一行一列得出的矩阵的行列式,其绝对值为生成树的个数
\\
因此,对于给定的图 G,若要求其生成树个数,可以先求其基尔霍夫矩阵,然后随意取其任意一个 n-1 阶行列式,然后求出行列式的值,其绝对值就是这个图中生成树的个数。
\\
度数矩阵 D[G]:当$ i\neq j$ 时,$D[i][j]=0$,当$ i=j$ 时,$D[i][i] = degree(v_i)$
\\
邻接矩阵 A[G]:当 $v_i$、$v_j$ 有边连接时,$A[i][j]=1$,当 $v_i$、$v_j$ 无边连接时,$A[i][j]=0$
\\
基尔霍夫矩阵(Kirchhoff) K[G]:也称拉普拉斯算子,其定义为$K[G]=D[G]-A[G]$,即:$K[i][j]=D[i][j]-A[i][j]$
\\