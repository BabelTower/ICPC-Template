\documentclass[10pt,a4paper]{article}
\usepackage{ctex} % 支持中文
%\usepackage{listings}
%\usepackage{zh_CN-Adobefonts_external}
\usepackage{xeCJK}
\usepackage{color}
\usepackage{amsmath, amsthm}
\usepackage{listings,xcolor}
\usepackage{geometry} % 设置页边距
\usepackage{fontspec}
\usepackage{graphicx}
\usepackage[colorlinks]{hyperref}
\usepackage{setspace}
\usepackage{fancyhdr} % 自定义页眉页脚

%\setsansfont{Consolas} % 设置英文字体
%\setmonofont[Mapping={}]{Consolas} % 英文引号之类的正常显示,相当于设置英文字体

\linespread{1.2}

\title{ACM template}
\author{fangzeyu @ ZJUT}

\definecolor{dkgreen}{rgb}{0,0.6,0}
\definecolor{gray}{rgb}{0.5,0.5,0.5}
\definecolor{mauve}{rgb}{0.58,0,0.82}

\pagestyle{fancy}

\lhead{\CJKfamily{kai} Zhejiang University of Technology} %以下分别为左中右的页眉和页脚
\chead{}

\rhead{\CJKfamily{kai} 第 \thepage 页}
\lfoot{} 
\cfoot{\thepage}
\rfoot{}
\renewcommand{\headrulewidth}{0.4pt} 
\renewcommand{\footrulewidth}{0.4pt}
%\geometry{left=2.5cm,right=3cm,top=2.5cm,bottom=2.5cm} % 页边距
\geometry{left=3.18cm,right=3.18cm,top=2.54cm,bottom=2.54cm}
\setlength{\columnsep}{30pt}

\makeatletter

\makeatother

\lstset{
    language    = c++,
    numbers     = left,
    numberstyle={                               % 设置行号格式
        \small
        \color{black}
        \fontspec{PingFangSC-Regular}
    },
	commentstyle = \color[RGB]{0,128,0}\bfseries, %代码注释的颜色
	keywordstyle={                              % 设置关键字格式
        \color[RGB]{40,40,255}
        \fontspec{PingFangSC-Regular}
        \bfseries
    },
	stringstyle={                               % 设置字符串格式
        \color[RGB]{128,0,0}
        \fontspec{PingFangSC-Regular}
        \bfseries
    },
	basicstyle={                                % 设置代码格式
        \fontspec{PingFangSC-Regular}
        \small\ttfamily
    },
	emphstyle=\color[RGB]{112,64,160},          % 设置强调字格式
    breaklines=true,                            % 设置自动换行
    tabsize     = 4,
    frame       = single,%主题
    columns     = fullflexible,
    rulesepcolor = \color{red!20!green!20!blue!20}, %设置边框的颜色
    showstringspaces = false, %不显示代码字符串中间的空格标记
	escapeinside={\%*}{*)},
}

\begin{document}
\maketitle
\newpage 

\tableofcontents

\newpage\section{基础}
\subsection{快读}
\lstinputlisting{基础/read.cpp}
\subsection{bitset}
\begin{spacing}{1.5}
\paragraph{bitset用法}
1、 二进制数的低位存储在bitset的低位,高位用0填充。

\end{spacing}
\lstinputlisting{基础/bitset.cpp}
\subsection{int128}
\lstinputlisting{基础/int128.cpp}
\subsection{vector去重}
\lstinputlisting{基础/vector去重.cpp}
\section{动态规划}
\subsection{背包DP}
\subsubsection{树形背包}
\lstinputlisting{动态规划/背包DP/树形背包.cpp}
\subsection{概率DP}
\subsubsection{概率DP}
\begin{spacing}{1.5}
\paragraph{概率 — 期望系统的定义}~{}
\par
概率 — 期望系统是一个带权的有向图。这个图中的点代表一个事件,而如果点 A 与点 B 之间有一条权为 p 的边,就表示 A 发生后, B 紧接着发生的概率是 p 。
初始的时候,有一个点(叫做初始点)代表的事件发生了,其他事件根据概率依次发生,每次只发生一个。
求其他各个事件发生次数的期望。记时间 A 的发生次数期望为 $E_{A}$ , A 到 B 的边权为 $P_{AB}$ \par
\textbf{限制}:
\begin{itemize}
    \item 对任意的 AB , $P_{AB} \leq 1$
    \item 对于任意点 A , $ \sum_{(A,B)\in E}{P_{AB}} \leq 1$,且对于系统中的所有点,至少有一个点使等号不成立。如果等号都成立的话这些事件将无穷无尽的发生下去,而概率 — 期望系统则变得没有意义(此时期望或者是无穷大,或者是 0 )。
    \item 不能有指向初始点的边,这是因为求解时我们把初始顶点的概率设为 1 。但是如果真的有这样的边,可以添加一个假点作为初始点,这个假点到真正的初始点有一条概率为 1 的边。
\end{itemize}
\paragraph{概率 — 期望系统的求解}~{}
\par
可以根据是否为DAG图,将问题分为两类。\par
有向无环图的概率 — 期望系统。这种系统是很简单的,因为它没有后效性,所以可以通过动态规划的方法在 O(E) 的时间内解决。许多使用动态规划解决的概率 — 期望问题都是基于这类系统的。\par
在有些问题中,我们需要解决更一般的概率 — 期望系统。这时图中含有圈,因而造成了后效性。\par

\paragraph{高斯消元解决后效性概率DP}~{}
\par
$$E_{A} = \sum_{(B,A)\in E}P_{BA} E_{B}$$ 
求线性方程组的解,我们更常用的稳定算法是高斯消元法,完全可以在这里使用。这样就得到了一种稳定而精确的解法:
首先根据概率 — 期望系统建立方程组,然后用高斯消元法去解,得到的结果就是我们要求的期望。这种算法的时间复杂度是 $O(n^3)$ 。

\paragraph{一种可以消除后效性的特例}~{}
\par
\textbf{限制}:
\begin{itemize}
    \item 图为有向无环图(线性递推、树),非DAG图需要用Tarjan算法缩点
    \item 存在无后效性可以转移的节点,即只从前置状态转移而来
    \item 每个状态只从常数个数的状态转移而来
\end{itemize}\par
\textbf{解法:}\par
根据DAG图的逆拓扑序,将所有节点的转移方程依次变为无后效性的。原理等价于高斯消元,
\end{spacing}
\lstinputlisting{动态规划/概率DP/概率DP.cpp}
\subsubsection{树上概率DP}
\lstinputlisting{动态规划/概率DP/树上概率DP.cpp}
\subsection{数位DP}
\subsubsection{模版}
\lstinputlisting{动态规划/数位DP/模版.cpp}
\subsection{树形DP}
\subsubsection{树上背包}
\lstinputlisting{动态规划/树形DP/树上背包.cpp}
\subsubsection{换根DP}
\lstinputlisting{动态规划/树形DP/换根DP.cpp}
\subsection{动态DP}
\begin{spacing}{1.5}
\paragraph{广义矩阵乘法}~{}
\par

定义广义矩阵乘法 $A\times B=C$ 为:

$$
C_{i,j}=\max_{k=1}^{n}(A_{i,k}+B_{k,j})
$$

相当于将普通的矩阵乘法中的乘变为加,加变为 $\max$ 操作。\par

同时广义矩阵乘法满足结合律,所以可以使用矩阵快速幂。\par

\paragraph{不带修改操作}~{}
\par

令 $f_{i,0}$ 表示不选择 $i$ 的最大答案,$f_{i,1}$ 表示选择 $i$ 的最大答案。\par

则有 DP 方程:

$$
\begin{cases}f_{i,0}=\sum_{son}\max(f_{son,0},f_{son,1})\\f_{i,1}=w_i+\sum_{son}f_{son,0}\end{cases}
$$

答案就是 $\max(f_{root,0},f_{root,1})$.\par

\paragraph{带修改操作}~{}
\par


设 $g_{i,0}$ 表示不选择 $i$ 且只允许选择 $i$ 的轻儿子所在子树的最大答案,$g_{i,1}$ 表示选择 $i$ 的最大答案,$son_i$ 表示 $i$ 的重儿子。\par

假设我们已知 $g_{i,0/1}$ 那么有 DP 方程:

$$
\begin{cases}f_{i,0}=g_{i,0}+\max(f_{son_i,0},f_{son_i,1})\\f_{i,1}=g_{i,1}+f_{son_i,0}\end{cases}
$$

答案是 $\max(f_{root,0},f_{root,1})$.\par

可以构造出矩阵:

$$
\begin{bmatrix}
g_{i,0} & g_{i,0}\\
g_{i,1} & -\infty
\end{bmatrix}\times 
\begin{bmatrix}
f_{son_i,0}\\f_{son_i,1}
\end{bmatrix}=
\begin{bmatrix}
f_{i,0}\\f_{i,1}
\end{bmatrix}
$$

注意,我们这里使用的是广义乘法规则。\par

可以发现,修改操作时只需要修改 $g_{i,1}$ 和每条往上的重链即可。\par

\paragraph{具体思路}~{}
\par
\begin{itemize}
    \item DFS 预处理求出 $f_{i,0/1}$ 和 $g_{i,0/1}$.
    \item 对这棵树进行树剖(注意,因为我们对一个点进行询问需要计算从该点到该点所在的重链末尾的区间矩阵乘,所以对于每一个点记录 $End_i$ 表示 $i$ 所在的重链末尾节点编号),每一条重链建立线段树,线段树维护 $g$ 矩阵和 $g$ 矩阵区间乘积。
    \item 修改时首先修改 $g_{i,1}$ 和线段树中 $i$ 节点的矩阵,计算 $top_i$ 矩阵的变化量,修改到 $fa_{top_i}$ 矩阵。
    \item 查询时就是 1 到其所在的重链末尾的区间乘,最后取一个 $\max$ 即可。
\end{itemize}
\end{spacing}
\lstinputlisting{动态规划/动态DP.cpp}
\subsection{插头DP}
\subsubsection{路径模型}
\lstinputlisting{动态规划/插头DP/路径模型.cpp}
\subsubsection{多条回路}
\lstinputlisting{动态规划/插头DP/多条回路.cpp}
\subsection{状压DP}
\subsubsection{状压DP}
\begin{spacing}{1.5}
技巧:
1. 用位运算优化判断,如判合法,或者有无相邻的1可以用式子 $s1\&(s1<<1)$

空间优化:
1. 使用滚动数组,而非swap,只要记录当前状态和上一个状态,节省一维空间

时间优化:
1. 先将所有合法状态过滤出来

算法:
1. 初始化
2. 状态转移
3. 统计答案
\end{spacing}
\lstinputlisting{动态规划/状压DP/状压DP.cpp}
\subsection{四边形不等式优化}
\subsubsection{总结}
\begin{spacing}{1.5}
\paragraph{一、DP时间复杂度}
$$
\text{时间复杂度} = \text{状态总数} \times \text{每个状态转移的状态数} \times \text{每次状态转移的时间}
$$

\paragraph{二、各类优化方式}
\subparagraph{1. 决策单调性}~{}
\par

四边形不等式的性质在一类 1D1D 动态规划中得出决策单调性,从而优化状态转移的复杂度。\par

\textbf{1D1D动态规划:} DP方程形如 $f_{r} = \min_{l=1}^{r-1}\{f_{l}+w(l,r)\}\quad\left(1 \leq r \leq n\right)$ 
,状态数为 $O(n)$ ,每一步决策量为 $O(n)$ 。\par

\textbf{决策单调性:} 设 $k_i$ 表示 $f[i]$ 转移的最优决策点,那么决策单调性可描述为 
$\forall i \leq j, k_i \leq k_j$。也就是说随着i的增大,所找到的最优决策点是递增态(非严格递增)。\par

定理: 若函数 $w(l,r)$ 满足四边形不等式,记 $h_{l,r}=f_l+w(l,r)$ 表示从 $l$ 转移过来的状态 $r$ , 
$k_{r}=\min\{l|f_{r}=h_{l,r}\}$ 表示最优决策点,则有$$\forall r_1 \leq r_2:k_{r_1} \leq k_{r_2}$$ \par

\textbf{四边形不等式:} 如果对于任意 $l_1\leq l_2 \leq r_1 \leq r_2$ ,
均有 $w(l_1,r_1)+w(l_2,r_2) \leq w(l_1,r_2) + w(l_2,r_1)$ 成立,
则称函数 $w$ 满足四边形不等式(简记为“交叉小于包含”)。
若等号永远成立,则称函数 $w$ 满足四边形恒等式。\par

我们根据决策单调性只能得出每次枚举 $l$ 时的下界,而无法确定其上界。因此,简单实现该状态转移方程仍然无法优化最坏时间复杂度。\par

\subparagraph{2. 决策单调性(分治)}~{}
\par

先考虑一种简单的情况,转移函数的值在动态规划前就已完全确定。即如下所示状态转移方程:

$$
f_{r} = \min_{l=1}^{r-1}w(l,r) \qquad\left(1 \leq r \leq n\right)
$$

在这种情况下,我们定义过程 $\textsf{DP}(l, r, k_l, k_r)$ 表示求解 $f_{l}\sim f_{r}$ 的状态值,
并且已知这些状态的最优决策点必定位于 $[k_l, k_r]$ 中,然后使用分治算法如 \textbf{单调性决策(分治)} 中所诉。 \par

使用递归树的方法,容易分析出该分治算法的复杂度为 $O(n\log n)$ ,
因为递归树每一层的决策区间总长度不超过 $2n$ ,而递归层数显然为 $O(\log n)$ 级别。 \par

\subparagraph{3. 决策单调性(二分栈)}~{}
\par

处理一般情况,即转移函数的值是在动态规划的过程中按照一定的拓扑序逐步确定的。
此时我们需要改变思维方式,由“确定一个状态的最优决策”转化为“确定一个决策是哪些状态的最优决策”。 \par

用栈维护单调的决策点,二分找到是哪些状态的最优决策,时间复杂度为 $O(n\log n)$。 \par

\subparagraph{4. 区间类(2D1D)动态规划}~{}
\par

在区间类动态规划(如石子合并问题)中,我们经常遇到以下形式的 2D1D 状态转移方程:

$$
f_{l,r} = \min_{k=l}^{r-1}\{f_{l,k}+f_{k+1,r}\} + w(l,r)\qquad\left(1 \leq l \leq r \leq n\right)
$$

直接简单实现状态转移,总时间复杂度将会达到 $O(n^3)$ ,但当函数 $w(l,r)$ 满足一些特殊的性质时,我们可以利用决策的单调性进行优化。\par

\textbf{区间包含单调性:}如果对于任意 $l \leq l' \leq r' \leq r$ ,均有 $w(l',r') \leq w(l,r)$ 成立,则称函数 $w$ 对于区间包含关系具有单调性。 \par
\textbf{四边形不等式:}如果对于任意 $l_1\leq l_2 \leq r_1 \leq r_2$ ,均有 $w(l_1,r_1)+w(l_2,r_2) \leq w(l_1,r_2) + w(l_2,r_1)$ 成立,
则称函数 $w$ 满足四边形不等式(简记为“交叉小于包含”)。若等号永远成立,则称函数 $w$ 满足 四边形恒等式 。\par

引理 1 :若满足关于区间包含的单调性的函数 $w(l, r)$ 满足四边形不等式,则状态 $f_{l,r}$ 也满足四边形不等式。\par

定理 1 :若状态 $f$ 满足四边形不等式,记 $m_{l,r}=\min\{k:f_{l,r} = g_{k,l,r}\}$ 表示最优决策点,则有

$$
m_{l,r-1} \leq m_{l,r} \leq m_{l+1,r}
$$

因此,如果在计算状态 $f_{l,r}$ 的同时将其最优决策点 $m_{l,r}$ 记录下来,那么我们对决策点 $k$ 的总枚举量将降为

$$
\sum_{1\leq l<r\leq n} m_{l+1,r} - m_{l,r-1} = \sum_{i=1}^n m_{i,n} - m_{1,i}\leq n^2
$$

\subparagraph{5. 满足四边形不等式的函数类}~{}
\par

为了更方便地证明一个函数满足四边形不等式,我们有以下几条性质:

性质 1 :若函数 $w_1(l,r),w_2(l,r)$ 均满足四边形不等式(或区间包含单调性),则对于任意 $c_1,c_2\geq 0$ ,函数 $c_1w_1+c_2w_2$ 也满足四边形不等式(或区间包含单调性)。

性质 2 :若存在函数 $f(x),g(x)$ 使得 $w(l,r) = f(r)-g(l)$ ,则函数 $w$ 满足四边形恒等式。当函数 $f,g$ 单调增加时,函数 $w$ 还满足区间包含单调性。

性质 3 :设 $h(x)$ 是一个单调增加的凸函数,若函数 $w(l,r)$ 满足四边形不等式并且对区间包含关系具有单调性,则复合函数 $h(w(l,r))$ 也满足四边形不等式和区间包含单调性。

性质 4 :设 $h(x)$ 是一个凸函数,若函数 $w(l,r)$ 满足四边形恒等式并且对区间包含关系具有单调性,则复合函数 $h(w(l,r))$ 也满足四边形不等式。
\end{spacing}
\subsubsection{1D1D分治}
\lstinputlisting{动态规划/四边形不等式优化/分治.cpp}
\subsubsection{1D1D二分栈}
\lstinputlisting{动态规划/四边形不等式优化/二分栈.cpp}
\subsubsection{区间类2D1D}
\lstinputlisting{动态规划/四边形不等式优化/区间类2D1D.cpp}
\subsection{单调栈单调队列优化}
\subsubsection{一维}
\lstinputlisting{动态规划/单调栈单调队列优化/一维.cpp}
\subsubsection{二维}
\lstinputlisting{动态规划/单调栈单调队列优化/二维.cpp}
\subsection{矩阵优化}
\subsubsection{矩阵乘法}
\lstinputlisting{动态规划/矩阵优化/矩阵乘法.cpp}
\subsubsection{广义矩阵乘法}
\lstinputlisting{动态规划/矩阵优化/广义矩阵乘法.cpp}
\subsection{GarsiaWachs算法}
\lstinputlisting{动态规划/GarsiaWachs算法.cpp}
\section{字符串}
\subsection{KMP}
\lstinputlisting{字符串/KMP.cpp}
\subsection{扩展KMP}
\lstinputlisting{字符串/exKMP.cpp}
\subsection{最小表示法}
\lstinputlisting{字符串/最小表示法.cpp}
\subsection{哈希}
\lstinputlisting{字符串/hash.cpp}
\subsection{马拉车}
\lstinputlisting{字符串/Manacher.cpp}
\subsection{字典树}
\subsubsection{Trie}
\lstinputlisting{字符串/字典树/Trie.cpp}
\subsubsection{01-Trie}
\lstinputlisting{字符串/字典树/01-Trie.cpp}
\subsection{后缀数组}
\subsubsection{总结}
\begin{spacing}{1.5}
\paragraph{应用}~{}
\par
\subparagraph{比较一个字符串的两个子串的大小关系}~{}
\par
假设需要比较的是 $A=S[a..b]$ 和 $B=S[c..d]$ 的大小关系。\par
若 $lcp(a, c)\ge\min(|A|, |B|)$,$A<B\iff |A|<|B|$。\par
否则,$A<B\iff rk[a]< rk[b]$。\par

\subparagraph{不同子串的数目}~{}
\par
子串就是后缀的前缀,所以可以枚举每个后缀,计算前缀总数,再减掉重复。\par
“前缀总数”其实就是子串个数,为 $n(n+1)/2$。\par
如果按后缀排序的顺序枚举后缀,每次新增的子串就是除了与上一个后缀的 LCP 剩下的前缀。这些前缀一定是新增的,否则会破坏 $lcp(sa[i],sa[j])=\min\{height[i+1..j]\}$ 的性质。只有这些前缀是新增的,因为 LCP 部分在枚举上一个前缀时计算过了。\par
所以答案为:$\frac{n(n+1)}{2}-\sum\limits_{i=2}^nheight[i]$\par
\end{spacing}
\subsubsection{倍增}
\lstinputlisting{字符串/后缀数组/da.cpp}
\subsubsection{DC3}
\lstinputlisting{字符串/后缀数组/dc3.cpp}
\subsection{单串SAM}
\begin{spacing}{1.5}
\paragraph{概述}~{}
\par
直观上,字符串的 Suffix Automation(SAM) 可以理解为给定字符串的 \textbf{所有子串} 的压缩形式。
值得注意的事实是,SAM 将所有的这些信息以高度压缩的形式储存。对于一个长度为 $n$ 的字符串,它的空间复杂度仅为 $O(n)$ 。
此外,构造 SAM 的时间复杂度仅为 $O(n)$ 。准确地说,一个 SAM 最多有 $2n-1$ 个节点和 $3n-4$ 条转移边。 \par

\paragraph{性质}~{}
\par
裸的后缀自动机仅仅是一个可以接收子串的自动机,在它的状态结点上维护的性质才是解题的关键。\par

子串可以根据它们结束的位置 $endpos$ 被划分为多个等价类, SAM 由初始状态 $t_0$ 和与每一个 $endpos$ 等价类对应的每个状态组成。\par

一个构造好的 SAM 实际上包含了两个图:\par
\begin{enumerate}
\item 由 $next$ 数组组成的 $DAG$ 图;\par
\item 由 $link$ 指针构成的 $parent$ 树。\par
\end{enumerate}
SAM 的状态结点包含了很多重要的信息:\par
\begin{itemize}
\item $maxlen$:即代码中 $len$ 变量,它表示该状态能够接受的最长的字符串长度。\par
\item $minlen$:表示该状态能够接受的最短的字符串长度。实际上等于该状态的 $link$ 指针指向的结点的 $len+1$。\par
\item $maxlen-minlen+1$:表示该状态能够接受的不同的字符串数。\par
\item $right$:即 结束位置的集合 $endpos-set$ 的个数,表示这个状态在字符串中出现了多少次,
该状态能够表示的所有字符串均出现过 $right$ 次。\par
\item $link$:$link$ 指向了一个能够表示当前状态表示的所有字符串的最长公共后缀的结点。
所有的状态的 $link$ 指针构成了一个 $parent$ 树,恰好是字符串的逆序的后缀树。\par
\item $parent$ 树的拓扑序:序列中第i个状态的子结点必定在它之后,父结点必定在它之前。\par
\end{itemize}
如果我们从任意状态 $v_0$ 开始顺着后缀链接遍历,总会到达初始状态 $t_0$ 。
这种情况下我们可以得到一个互不相交的区间 $[\operatorname{minlen}(v_i),\operatorname{len}(v_i)]$ 的序列,
且它们的并集形成了连续的区间 $[0,\operatorname{len}(v_0)]$ 。\par


\paragraph{拓展}~{}
\par
设字符串的长度为 $n$ ,考虑 $extend$ 操作中 $cur$ 变量的值,这个节点对应的状态是执行 $extend$ 操作时的当前字符串,即字符串的一个前缀,每个前缀有一个终点。
这样得到的 $n$ 个节点,对应了 $n$ 个不同的 终点 。设第 $i$ 个节点为 $v_i$ ,对应的是 $S_{1 \ldots i}$ ,终点是 $i$ 。姑且把这些节点称之为“终点节点”。\par

考虑给 SAM 赋予树形结构,树的根为 0,且其余节点 $v$ 的父亲为 $\operatorname{link}(v)$ 。则这棵树与原 SAM 的关系是:\par
\begin{itemize}
\item 每个节点的终点集合等于其 子树 内所有终点节点对应的终点的集合。\par
\end{itemize}
在此基础上可以给每个节点赋予一个最长字符串,是其终点集合中 任意 一个终点开始 往前 取 len 个字符得到的字符串。每个这样的字符串都一样,且 len 恰好是满足这个条件的最大值。\par

这些字符串满足的性质是:\par
\begin{itemize}
\item 如果节点 A 是 B 的祖先,则节点 A 对应的字符串是节点 B 对应的字符串的 后缀 。\par
\end{itemize}
这条性质把字符串所有前缀组成了一棵树,且有许多符合直觉的树的性质。
例如, $S_{1 \ldots p}$ 和 $S_{1 \ldots q}$ 的最长公共后缀对应的字符串就是 $v_p$ 和 $v_q$ 对应的 LCA 的字符串。
实际上,这棵树与将字符串 $S$ 翻转后得到字符串的压缩后缀树结构相同。\par

\paragraph{构造}~{}
\par
构造SAM是在线算法,逐个加入字符串的字符过程中,每一步对应的维护SAM。\par

为了保证线性的空间复杂度,状态节点中将只保存 $\operatorname{len}$ 和 $\operatorname{link}$ 的值和每个状态的转移列表。\par

一开始 SAM 只包含一个状态 $t_0$ ,编号为 $0$ (其它状态的编号为 $1,2,\ldots$ )。
为了方便,对于状态 $t_0$ 我们指定 $\operatorname{len}=0$ 、 $\operatorname{link}=-1$ ( $-1$ 表示虚拟状态)。\par

现在,任务转化为实现给当前字符串添加一个字符 $c$ 的过程。算法流程如下:
\begin{enumerate}
\item 令 $\textit{last}$ 为添加字符 $c$ 之前,整个字符串对应的状态(一开始我们设 $\textit{last}=0$ ,算法的最后一步更新 $\textit{last}$ )。
\item 创建一个新的状态 $\textit{cur}$ ,并将 $\operatorname{len}(\textit{cur})$ 赋值为 $\operatorname{len}(\textit{last})+1$ ,在这时 $\operatorname{link}(\textit{cur})$ 的值还未知。
\item 现在我们按以下流程进行(从状态 $\textit{last}$ 开始)。如果还没有到字符 $c$ 的转移,我们就添加一个到状态 $\textit{cur}$ 的转移,遍历后缀链接。
如果在某个点已经存在到字符 $c$ 的转移,我们就停下来,并将这个状态标记为 $p$ 。
\item 如果没有找到这样的状态 $p$ ,我们就到达了虚拟状态 $-1$ ,我们将 $\operatorname{link}(\textit{cur})$ 赋值为 $0$ 并退出。
\item 假设现在我们找到了一个状态 $p$ ,其可以通过字符 $c$ 转移。我们将转移到的状态标记为 $q$ 。
\item 现在我们分类讨论两种状态,要么 $\operatorname{len}(p) + 1 = \operatorname{len}(q)$ ,要么不是。
\item 如果 $\operatorname{len}(p)+1=\operatorname{len}(q)$ ,我们只要将 $\operatorname{link}(\textit{cur})$ 赋值为 $q$ 并退出。
\item 否则就会有些复杂。需要 复制 状态 $q$ :我们创建一个新的状态 $\textit{clone}$ ,复制 $q$ 的除了 $\operatorname{len}$ 的值以外的所有信息(后缀链接和转移)。
我们将 $\operatorname{len}(\textit{clone})$ 赋值为 $\operatorname{len}(p)+1$ 。  
复制之后,我们将后缀链接从 $\textit{cur}$ 指向 $\textit{clone}$ ,也从 $q$ 指向 $\textit{clone}$ 。  
最终我们需要使用后缀链接从状态 $p$ 往回走,只要存在一条通过 $p$ 到状态 $q$ 的转移,就将该转移重定向到状态 $\textit{clone}$ 。
\item 以上三种情况,在完成这个过程之后,我们将 $\textit{last}$ 的值更新为状态 $\textit{cur}$ 。
\end{enumerate}

\textbf{标记终止状态}在构造完完整的 SAM 后找到所有的终止状态。\par
为此,我们从对应整个字符串的状态(存储在变量 $\textit{last}$ 中),遍历它的后缀链接,直到到达初始状态。我们将所有遍历到的节点都标记为终止节点。
容易理解这样做我们会准确地标记字符串 $s$ 的所有后缀,这些状态都是终止状态。


\end{spacing}
\lstinputlisting{字符串/后缀自动机.cpp}
\subsection{多串SAM}
\begin{spacing}{1.5}
\paragraph{概述}~{}
\par
后缀自动机 (suffix automaton, SAM) 是用于处理单个字符串的子串问题的强力工具。\par
而广义后缀自动机 (General Suffix Automaton) 则是将后缀自动机整合到字典树中来解决对于多个字符串的子串问题\par

\paragraph{实现}~{}
\par
\begin{itemize}
    \item 由于整个 $BFS$ 的过程得到的顺序,其父节点始终在变化,所以并不需要保存 `last` 指针。
    \item 插入操作中,`int cur = next[last][c];` 与正常后缀自动机的 `int cur = tot++;` 有差异,因为我们插入的节点已经在树型结构中完成了,所以只需要直接获取即可
    \item 在 $clone$ 后的数据拷贝中,有这样的判断 `next[clone][i] = len[next[q][i]] != 0 ? next[q][i] : 0;` 这与正常的后缀自动机的直接赋值 `next[clone][i] = next[q][i];` 有一定差异,此次是为了避免更新了 `len` 大于当前节点的值。由于数组中 `len` 当且仅当这个值被 $BFS$ 遍历并插入到后缀自动机后才会被赋值
\end{itemize}

\end{spacing}
\lstinputlisting{字符串/多串SAM.cpp}
\subsection{后缀树}
\lstinputlisting{字符串/后缀树.cpp}
\subsection{Lyndon分解}
\begin{spacing}{1.5}
\paragraph{Lyndon 串:}对于字符串 ,如果 的字典序严格小于 的所有后缀的字典序,我们称 是简单串,或者 Lyndon 串。举一些例子, `a` , `b` , `ab` , `aab` , `abb` , `ababb` , `abcd` 都是 Lyndon 串。当且仅当 的字典序严格小于它的所有非平凡的循环同构串时, 才是 Lyndon 串。
\paragraph{Lyndon 分解:}串的 Lyndon 分解记为$s=w_1 w_2 ...w_n$ ,其中所有$w_i$为简单串,并且他们的字典序按照非严格单减排序 ,即 $w_1{\geq}w_2{\geq}...{\geq}w_n$。可以发现,这样的分解存在且唯一。
\end{spacing}
\lstinputlisting{字符串/Lyndon分解.cpp}
\section{数据结构}
\subsection{ST表}
\lstinputlisting{数据结构/ST表.cpp}
\subsection{线段树}
\lstinputlisting{数据结构/线段树.cpp}
\subsection{二维线段树}
\lstinputlisting{数据结构/二维线段树.cpp}
\subsection{主席树}
\lstinputlisting{数据结构/主席树.cpp}
\subsection{主席树}
\lstinputlisting{数据结构/主席树.cpp}
\subsection{线性基}
\lstinputlisting{数据结构/线性基.cpp}
\subsection{舞蹈链}
\lstinputlisting{数据结构/舞蹈链.cpp}
\subsection{划分树}
\lstinputlisting{数据结构/划分树.cpp}
\subsection{单调栈}
\begin{spacing}{1.5}
\paragraph{单调栈的应用}~{}
\par
单调栈只能维护栈内元素,以某一特征递增或递减,而其应用无法离开这条性质。\par
举个简单的例子,存在长度为n的数组$a[i]$,现要求对$i\in [1,n]$,输出$\sum_{j=1}^{i-1}Min(a[k], k\in [j,i])$。\par
朴素的想法是暴力向前扫一遍,维护区间最小值,这样的复杂度是$O(n^2)$的。\par
依靠单调栈优化,实现方法是维护单调栈自顶向下递减,对栈内每个元素统计其val和cnt,val意为对答案的贡献,cnt为贡为val的下标个数,栈内所有元素$\sum cnt*val$即为答案。\par
\end{spacing}
\subsection{并查集}
\subsubsection{带权并查集}
\lstinputlisting{数据结构/并查集/带权并查集.cpp}
\subsubsection{可撤销并查集}
\lstinputlisting{数据结构/并查集/可撤销并查集.cpp}
\subsubsection{可撤销种类并查集}
\lstinputlisting{数据结构/并查集/可撤销种类并查集.cpp}
\section{计算几何}
\subsection{判两条线段相交}
\lstinputlisting{计算几何/判两条线段相交.cpp}
\section{数论}
\subsection{埃式筛法}
\lstinputlisting{数论/埃式筛法.cpp}
\subsection{分数类}
\lstinputlisting{数论/分数类.cpp}
\subsection{三分}
\lstinputlisting{数论/三分.cpp}
\subsection{组合数}
\lstinputlisting{数论/组合数.cpp}
\subsection{exgcd}
\lstinputlisting{数论/exgcd.cpp}
\subsection{高斯消元}
\lstinputlisting{数论/高斯消元.cpp}
\subsection{计数}
\begin{spacing}{1.5}
\paragraph{分配}~{}
\par
\begin{itemize}
    \item n个元素分配到m个固定大小的容器里的方案数($cnt[i]$为容器i的大小):$\frac{n!}{\prod_{i=1}^{m}cnt[i]!}$
\end{itemize}
\paragraph{树的种类数}~{}
\par
\begin{itemize}
    \item Cayley公式:一个完全图$K_n$有$n^{n-2}$棵生成树,n个节点的带标号的无根树有$n^{n-2}$个
    \item n个结点构成的二叉树种类个数:$b_n=\frac{C^{n}_{2n}}{n+1}$
\end{itemize}
\end{spacing}
\subsection{数列}
\subsubsection{oeis}
\lstinputlisting{数论/数列/oeis.cpp}
\subsubsection{Bell数}
\lstinputlisting{数论/数列/bell.cpp}
\subsubsection{Catalan数}
\lstinputlisting{数论/数列/Catalan.cpp}
\subsubsection{超级Catalan数}
\lstinputlisting{数论/数列/super-Catalan.cpp}
\subsubsection{stirling}
\lstinputlisting{数论/数列/stirling.cpp}
\section{图论}
\subsection{有向图判环}
\begin{spacing}{1.5}
对一个有向图的节点进行拓扑排序,可以用来判断该有向图是否成环,有环则无拓扑序列,无环则有。
\end{spacing}
\subsection{HK算法}
\lstinputlisting{图论/HK算法.cpp}
\subsection{点分治}
\lstinputlisting{图论/点分治.cpp}
\subsection{树链剖分}
\lstinputlisting{图论/树链剖分.cpp}
\subsection{矩阵树定理}
\lstinputlisting{图论/矩阵树定理.cpp}
\subsection{一般图最大匹配}
\lstinputlisting{图论/一般图最大匹配.cpp}
\subsection{最近公共祖先}
\lstinputlisting{图论/最近公共祖先.cpp}
\subsection{最小树形图}
\lstinputlisting{图论/最小树形图.cpp}
\subsection{二分图}
\begin{spacing}{1.5}
\paragraph{性质}~{}
\\
一般图:\\
1.对于不存在孤立点的图,|最大匹配|+|最小边覆盖|=|E|\\
2.|最大独立集|+|最小顶点覆盖|=|V|\\
二分图:\\
|最大匹配|=|最小顶点覆盖|\\
\end{spacing}
\subsection{判无向图是否为二分图}
\begin{spacing}{1.5}
判断无向图是否为二分图,等价于判无向图内是否有奇环,有两种方式:\\ 
1. 染色法dfs/bfs遍历整个图,对相邻节点染不同颜色,遇矛盾则说明不是二分图。\\ 
2. 用种类并查集实现,原理与染色法相似,但支持并查集的可持久化、可撤销、可删边等操作。\\ 
\end{spacing}
\subsection{KM算法}
\lstinputlisting{图论/KM算法.cpp}
\subsection{染色法判二分图}
\lstinputlisting{图论/染色法判二分图.cpp}
\subsection{匈牙利算法}
\lstinputlisting{图论/匈牙利算法.cpp}
\subsection{矩阵树定理}
\begin{spacing}{1.5}
\paragraph{矩阵树定理}~{}
\\
对于生成树的计数,一般采用矩阵树定理(Matrix-Tree 定理)来解决。
\\
Matrix-Tree 定理的内容为:对于已经得出的基尔霍夫矩阵,去掉其随意一行一列得出的矩阵的行列式,其绝对值为生成树的个数
\\
因此,对于给定的图 G,若要求其生成树个数,可以先求其基尔霍夫矩阵,然后随意取其任意一个 n-1 阶行列式,然后求出行列式的值,其绝对值就是这个图中生成树的个数。
\\
度数矩阵 D[G]:当$ i\neq j$ 时,$D[i][j]=0$,当$ i=j$ 时,$D[i][i] = degree(v_i)$
\\
邻接矩阵 A[G]:当 $v_i$、$v_j$ 有边连接时,$A[i][j]=1$,当 $v_i$、$v_j$ 无边连接时,$A[i][j]=0$
\\
基尔霍夫矩阵(Kirchhoff) K[G]:也称拉普拉斯算子,其定义为$K[G]=D[G]-A[G]$,即:$K[i][j]=D[i][j]-A[i][j]$
\\
\end{spacing}
\subsection{启发式合并}
\lstinputlisting{图论/启发式合并.cpp}
\subsection{2-SAT}
\lstinputlisting{图论/2-SAT.cpp}
\subsection{网络流}
\subsubsection{网络流}
\begin{spacing}{1.5}
\paragraph{最大流}~{}
\\
设点数为$n$,边数为$m$,那么Dinic算法的时间复杂度(在应用上面两个优化的前提下)是$O(n^{2}m)$,在稀疏图上效率和EK算法相当,但在稠密图上效率要比EK算法高很多。
\paragraph{最小割}~{}
\\
最大流最小割定理 $f(s,t)_{max}=c(s,t)_{min}$
\paragraph{费用流}~{}
\\

\end{spacing}
\subsubsection{最大费用流}
\lstinputlisting{图论/网络流/最大费用流.cpp}
\subsubsection{最大费用最大流}
\lstinputlisting{图论/网络流/最大费用最大流.cpp}
\subsubsection{zkw费用流}
\lstinputlisting{图论/网络流/zkw费用流.cpp}
\subsubsection{费用流}
\lstinputlisting{图论/网络流/费用流.cpp}
\subsubsection{dinic}
\lstinputlisting{图论/网络流/dinic.cpp}
\subsection{最短路}
\subsubsection{Floyd}
\begin{spacing}{1.5}
\paragraph{图的传递闭包}~{}
\par
已知一个有向图中任意两点之间是否有连边,要求判断任意两点是否连通。 \par
bitset 优化,复杂度可以到 $O(\frac{n^3}{w})$ \par
\paragraph{最小环}~{}
\par
给一个正权无向图,找一个最小权值和的环。\par
想一想这个环是怎么构,考虑环上编号最大的结点 u, $f[u-1][x][y]$ 和 (u,x), (u,y)共同构成了环。\par
在 Floyd 的过程中枚举 u,计算这个和的最小值即可。\par
时间复杂度为$O(n^3)$。\par
\end{spacing}
\lstinputlisting{图论/最短路/Floyd.cpp}
\subsubsection{Bellman-Ford}
\begin{spacing}{1.5}
\paragraph{应用}~{}
\par
给一张有向图,问是否存在负权环。\par
做法很简单,跑 Bellman-Ford 算法,如果有个点被松弛成功了 $n$ 次,那么就一定存在。\par
如果 $n-1$ 次之内算法结束了,就一定不存在。\par
在没有负权边时最好使用 Dijkstra 算法,在有负权边且题目中的图没有特殊性质时,若 SPFA 是标算的一部分,题目不应当给出 Bellman-Ford 算法无法通过的数据范围\par


\end{spacing}
\lstinputlisting{图论/最短路/Bellman-Ford.cpp}
\subsubsection{dijkstra}
\lstinputlisting{图论/最短路/dijkstra.cpp}
\subsubsection{bfs全源最短路径}
\lstinputlisting{图论/最短路/bfs全源最短路径.cpp}
\subsubsection{Johnson全源最短路径算法}
\lstinputlisting{图论/最短路/Johnson全源最短路径算法.cpp}
\section{杂项}
\subsection{随机化}
\subsubsection{模拟退火}
\lstinputlisting{杂项/随机化/模拟退火.cpp}
\end{document}